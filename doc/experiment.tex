\section{Experiment}
\paragraph{Data construction.}
We construct \emph{query} and \emph{context} sets from the Yelp Open Dataset using a multi-stage curation pipeline designed to ensure data quality and relevance.

\textbf{Curation Pipeline.}
First, we interactively select a target city and specific cuisine categories (e.g., ``Italian'', ``Mexican'') based on restaurant density.
To ensure the selected entities genuinely fit the target categories, we employ an automated LLM-based filtering step.
Candidates are ranked by \emph{richness} (total volume of review text) and processed by an LLM which estimates the probability of category fit (0--100\%) based on keyword evidence extracted from their reviews.
We retain only restaurants with a high probability of fit.

\textbf{Review Sampling.}
For each selected restaurant, we construct a document set of 20 reviews.
To capture diverse user sentiments, we stratify the sampling by star rating, selecting exactly 4 reviews from each star level (1--5 stars).
Within each stratum, reviews are selected by descending text length to maximize the information available for reasoning.

\textbf{Metadata Structure.}
Each query corresponds to a single restaurant and includes:
(i) \emph{Item Metadata}: structured fields such as name, location, categories, attributes, and hours; and
(ii) \emph{Review Data}: the set of 20 sampled reviews, including text, rating, date, and reviewer profile signals (e.g., elite status, friend count, year started).

\paragraph{Contexts (User Requests).}
Each context corresponds to a natural-language user request that specifies a set of constraints, which must be supported by evidence available in the query. 
Rather than assuming a single notion of difficulty, we organize contexts into groups that isolate distinct combinations of evidence sources and logical structures. 
Across groups, requests vary along two orthogonal axes: 
(1) the types of information sources they reference, including direct item metadata, computed metadata (e.g., hours or time-window constraints), review text, and reviewer or review metadata; and 
(2) the compositional structure of the constraints, ranging from flat Boolean chains to nested Boolean expressions with explicit scope.

Flat requests consist of sequential compositions of conditions connected by AND/OR operators (e.g., $A \,\mathrm{op}_1\, B \,\mathrm{op}_2\, C$), while nested requests require hierarchical Boolean structure (e.g., $A \land (B \lor C)$), testing the model’s ability to preserve operator precedence and propagate uncertainty across recursive compositions. 
Using this design, we construct 50 contexts grouped into 10 categories, each containing five requests, where each group isolates a specific combination of evidence sources and logical structure (simple vs.\ computed metadata, text, social signals; flat vs.\ balanced or skewed nesting).

\begin{itemize}
    \item \textbf{G01 (R00--R04): Flat / Simple Metadata.}
    Requests use only direct item metadata (e.g., attributes, categories) with flat AND/OR composition.

    \item \textbf{G02 (R05--R09): Flat / Review Text.}
    Requests rely exclusively on conditions grounded in review text, composed using flat AND/OR logic.

    \item \textbf{G03 (R10--R14): Flat / Computed Metadata.}
    Requests use computed item metadata requiring interpretation (e.g., hours, time-window constraints) with flat AND/OR logic.

    \item \textbf{G04 (R15--R19): Flat / Social Metadata.}
    Requests use only reviewer or review metadata (e.g., elite status, recency) with flat AND/OR logic.

    \item \textbf{G05 (R20--R24): Flat / Simple Metadata + Review Text.}
    Requests combine direct item metadata and review text conditions in flat AND/OR form.

    \item \textbf{G06 (R25--R29): Flat / Simple Metadata + Review Text + Social Metadata.}
    Requests integrate structured item metadata, unstructured review text, and social signals within a flat Boolean structure.

    \item \textbf{G07 (R30--R34): Nested (Balanced) / Simple Metadata + Review Text.}
    Requests use balanced nested Boolean expressions over simple item metadata and review text.

    \item \textbf{G08 (R35--R39): Nested (Balanced) / Computed Metadata + Social Metadata.}
    Requests use balanced nesting while combining computed item metadata and social signals.

    \item \textbf{G09 (R40--R44): Nested (Skewed) / Simple Metadata + Review Text.}
    Requests use skewed (left- or right-branching) nested structures over simple metadata and review text.

    \item \textbf{G10 (R45--R49): Nested (Skewed) / Computed Metadata + Social Metadata.}
    Requests combine skewed nesting with computed metadata and social signals.
\end{itemize}


